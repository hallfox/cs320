\documentclass[12pt]{article}
 
\usepackage[utf8]{inputenc}
\usepackage[margin=1in]{geometry} 
\usepackage{amsmath,amsthm,amssymb}
\usepackage{enumerate}
\usepackage{listings}
\usepackage{color}
\usepackage{courier}
 
\newcommand{\N}{\mathbb{N}}
\newcommand{\Z}{\mathbb{Z}}
 
\newenvironment{theorem}[2][Theorem]{\begin{trivlist}
\item[\hskip \labelsep{\bfseries #1}\hskip \labelsep{\bfseries #2.}]}{\end{trivlist}}
\newenvironment{lemma}[2][Lemma]{\begin{trivlist}
\item[\hskip \labelsep{\bfseries #1}\hskip \labelsep{\bfseries #2.}]}{\end{trivlist}}
\newenvironment{exercise}[2][Exercise]{\begin{trivlist}
\item[\hskip \labelsep{\bfseries #1}\hskip \labelsep{\bfseries #2.}]}{\end{trivlist}}
\newenvironment{problem}[2][Problem]{\begin{trivlist}
\item[\hskip \labelsep{\bfseries #1}\hskip \labelsep{\bfseries #2.}]}{\end{trivlist}}
\newenvironment{question}[2][Question]{\begin{trivlist}
\item[\hskip \labelsep{\bfseries #1}\hskip \labelsep{\bfseries #2.}]}{\end{trivlist}}
\newenvironment{corollary}[2][Corollary]{\begin{trivlist}
\item[\hskip \labelsep{\bfseries #1}\hskip \labelsep{\bfseries #2.}]}{\end{trivlist}}

\definecolor{codegreen}{rgb}{0,0.6,0}
\definecolor{codegray}{rgb}{0.5,0.5,0.5}
\definecolor{codepurple}{rgb}{0.58,0,0.82}
\definecolor{backcolour}{rgb}{0.95,0.95,0.92}

\lstdefinestyle{mystyle}{
    backgroundcolor=\color{backcolour},   
    commentstyle=\color{codegreen},
    keywordstyle=\color{magenta},
    numberstyle=\tiny\color{codegray},
    stringstyle=\color{codepurple},
    basicstyle=\footnotesize\ttfamily\color{black},
    breakatwhitespace=false,         
    breaklines=true,                 
    captionpos=b,                    
    keepspaces=true,                 
    numbers=left,                    
    numbersep=5pt,                  
    showspaces=false,                
    showstringspaces=false,
    showtabs=false,                  
    tabsize=2
}
 
\lstset{style=mystyle}
\renewcommand{\lstlistingname}{Code}

\begin{document}
 
\title{Comparing Cache Models}
\author{Taylor Foxhall\\
CS 320}
 
\maketitle
\section{Purpose}
\label{sec:label}

In this assignment we look to evaluate the effectiveness of different methods of
caching.
 
\section{Static Predictors}
\label{sec:label}

\subsection{Always Taken}
\label{subsec:label}

The algorithm for the static predictors is extremely simple. Essentially we want
the predictor to be the program's Yes Man. The entire algorithm is:

\begin{lstlisting}[language=C++, caption=Squirrel!, label=Always Taken]
std::string yes_predicitor(const std::vector<Prediction>& preds) {
    int taken = count_if(preds.begin(), preds.end(),
            [](Prediction p){return p.taken;});
    return std::to_string(taken) + "," + std::to_string(preds.size()) + ";";
}
\end{lstlisting}

The function is a simplified calcuation of how many times we take the
branch correctly. The acutal logic can be seen in the body of the lambda
function we pass to \lstinline|count_if|, which is a neat utility function from
\lstinline|<algorithm>|. Nothing too complex, just always predict we'll take the branch.

\subsection{Always Non-Taken}
\label{sec:label}

This algorithm differs trivially from the Always Taken strategy above. Instead
we always say no, and the only difference in the code above is that we
\lstinline|return !p.taken;| instead, and we get the exact opposite results.

In general, the results we get from the two methods are highly varied. Since the
results of the two are inversely related, if one performs well the other one
divebombs, making it a matter or whether or not you guessed more takens than
not-takens would occur.

\section{Dynamic Predictors}

\label{sec:label}

\subsection{1-Bit Bimodal Predictor}
\label{subsec:label}

The 1-Bit Predictor was extremely easy to implement, since indexing the history
table was a fairly simple algorithm to implement.

\begin{lstlisting}[language=C++, caption=Giving our predictor a brain]
// Table for 1-bit predictor
int size = table_sizes[i];
std::vector<bool> table(size, true);
int correct_preds = 0;
for (auto pred = preds.begin(); pred != preds.end(); pred++) {
    int pred_loc = pred->pc % size;
    if (pred->taken == table[pred_loc]) {
        correct_preds++;
    }
    else {
        table[pred_loc] = !table[pred_loc];
    }
}
\end{lstlisting}

By keeping track of some kind of history associated with the program counter
value, we began to see great improvements in accuracy. As we increased the table
size the accuracy made pretty good gains in accuracy, too. However, with each
table size increase we also saw diminishing returns, so at some point the
benefits will cap off.

\subsection{2-Bit Bimodal Predictor}
\label{subsec:label}

Now we increase the memory to 2 bits for each entry in the table and tells us
not to change our minds if we're wrong just once. This changes
our algorithm a little, but not greatly.

\begin{lstlisting}[languag=C++, caption=A few more conditions, but the same idea]
int size = table_sizes[i];
std::vector<unsigned char> table(size, 3);
int correct_preds = 0;
for (auto pred = preds.begin(); pred != preds.end(); pred++) {
    int pred_loc = pred->pc % size;
    if (pred->taken == table[pred_loc] > 1) {
        correct_preds++;
        if (table[pred_loc] == 2) {
            table[pred_loc] = 3;
        }
        else if (table[pred_loc] == 1) {
            table[pred_loc] = 0;
        }
    }
    else if (table[pred_loc] == 0 || table[pred_loc] == 2) {
        table[pred_loc] = 1;
    }
    else if (table[pred_loc] == 1 || table[pred_loc] == 3) {
        table[pred_loc] = 2;
    }
}
\end{lstlisting}

The results of this are pretty satisfying, for a small increase in memory we get
better accuracy for every table size and the accuracy gains after each increase
in size is more consistent and doesn't drop off as fast.

\subsection{Gshare Predictor}
\label{subsec:label}

The idea with the Gshare Predictor was to not only store state assosciated with
the program counter, but also remember the result of the last few branches we
found to help us more smartly figure out where to update the state of our 2 bits
from the 2-Bit Bimodal Predictor.

\begin{lstlisting}[language=C++, caption=Thanks Greg]
std::vector<unsigned char> table(2048, 3);
unsigned short greg = 0; // Greg will keep track of things for me
int correct_preds = 0;
for (auto pred = preds.begin(); pred != preds.end(); pred++) {
    int pred_loc = (pred->pc ^ greg) % 2048;
    if (pred->taken == table[pred_loc] > 1) {
        correct_preds++;
        if (table[pred_loc] == 2) {
            table[pred_loc] = 3;
        }
        else if (table[pred_loc] == 1) {
            table[pred_loc] = 0;
        }
    }
    else if (table[pred_loc] == 0 || table[pred_loc] == 2) {
        table[pred_loc] = 1;
    }
    else if (table[pred_loc] == 1 || table[pred_loc] == 3) {
        table[pred_loc] = 2;
    }
    greg = (greg << 1 | pred->taken) & r_mask;
}
\end{lstlisting}

Oddly enough, the results from this predictor get worse as we have a larger
local history. This may be because with a larger history the program is more
liable to ``confuse'' itself. This means that because a smaller global register
has fewer possible values it could take, which may translate to more consistent
table indexing.

\subsection{Tournament Predictor}
\label{subsec:label}
 
This predictor is the unholy love child of the 2-Bit Bimodal predictor and the
Gshare Predictor. Essentially it's trying to balance the losses from
``confusing'' the history table from Gshare's methods, but also compensating for
the diminishing returns from a large table history in the 2-Bit Bimodal model.
It keeps track of which method to use via another history table like in the
2-Bit model. The code is not shockingly different than what's already been
shown, so it's not been recopied here.

This predictor consistently beats the best results from both 2-Bit Bimodal and
Gshare. This seems like the most accurate version, but I have trouble believing
that managing 2 history tables and updating their state in a CPU doesn't inflict some
kind of significant overhead, either in complexity or speed.

\end{document}
